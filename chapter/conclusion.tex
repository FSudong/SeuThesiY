\chapter{总结与展望}
\section{本文工作总结}
本文首先描述了IT运维的研究背景及意义,分析了在实际运维场景中存在的问题:多源异构数据难以整合、运维知识表示不足和故障难以准确预知。随后,本文针对这些问题,对国内外研究现状展开了调研,并分析了已有方案的局限性。本文提出了自动构建组件-事件知识图谱、动态表示组件-事件知识图谱和结合知识图谱进行故障预测的一套算法模型。基于这些算法模型,本文设计并实现了基于知识图谱的IT运维辅助系统。本文完成的工作可分为以下4点:
% 首先,本文设计了事件特征,利用机器学习模型从硬件、软件、日志和指标时序数据中自动地构建组件-事件知识图谱。其次,本文设计了针对组件-事件知识图谱的动态表示学习模型,将实体表示分为语义表示和结构表示,实现了随着实体上下文变化动态地表示实体。最后,本文以上述表示学习模型作为嵌入层,利用双向记忆网络编码事件序列信息,再结合知识图谱进行了故障预测。本文不仅全面地获取并表示了系统多元异构信息,也结合了场景拓扑传播特征、先验知识、实时事件序列进行了表示学习和故障预测。本文完成的工作可分为以下4点:

(1)本文提出了全面整合多源异构数据,自动构建组件-事件知识图谱的方案。在判别事件因果关系时,本文引入了新的事件特征,提升了事件因果关系判别模型的效果。构成的知识图谱包含了高细粒度的多种信息,如软硬组件间关系、指标时序数据和日志数据,解决了多源异构数据难以整合的问题。

(2)本文提出了适配组件-事件知识图谱的动态表示学习模型。该模型将实体表示分为了语义表示和结构表示,语义表示通过实体文本信息获取,结构表示通过Attention-RGCN获取,实现了实体随上下文变化的动态表示,解决了运维知识表示不足的问题。

(3)本文提出了引入知识图谱的故障预测模型。该故障预测模型,利用各类故障对应的知识图谱识别事件序列中的关键信息,把最匹配事件序列的知识图谱作为预测结果,提高了预测结果的细粒度,增强了可解释性,解决了故障难以准确预知的问题。

(4)基于以上工作,本文设计并实现了基于知识图谱的IT运维辅助系统。该系统能够详细展示集群运行状态,并根据实时发生的事件序列和知识图谱预测故障,提醒运维人员及时采取应对措施,满足了实际的IT运维需求。

\section{未来工作展望}
本文深入调研IT运维难点及现有工作不足,提出了结合知识图谱、场景实时特征的基于知识图谱的IT运维辅助方案。但本文所作工作仍有继续优化的空间。

(1)需要在工业界项目上进一步验证效果。目前本系统已在开源的分布式应用train-ticket和sock-shop上取得了良好效果。但工业界应用如淘宝、饿了吗、滴滴等,相较本文使用的两个开源应用,在客户请求量、系统复杂度上都要高出较多量级。因此,本文需要寻求在工业级应用中进一步验证系统性能,并采取相对应的优化措施。

(3)自动生成新的知识图谱,拓展知识库。本文虽然已模拟了常见的十余种故障,但仍难以涵盖实际运行中可能出现的新故障,需要添加业务逻辑自动收集过往未出现的新故障数据并沉淀生成对应的组件-事件知识图谱。