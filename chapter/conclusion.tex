\chapter{系统设计与实现}
\section{本文工作总结}
本文首先分析了IT运维目前所遇到的难点、研究背景及研究意义,然后提出了基于知识图谱辅助IT运维的一套方案。首先,本文扩展了事件特征,利用机器学习模型从硬件、软件、日志和指标时序数据中自动地构建组件-事件知识图谱。其次,本文设计了针对组件-事件知识图谱的动态表示学习模型,将实体表示分为语义表示和结构表示,实现了随着实体上下文变化动态地表示实体。最后,本文以上述表示学习模型作为嵌入层,利用双向记忆网络编码事件序列信息,再结合知识图谱进行了故障预测。本文不仅全面地获取并表示了系统多元异构信息,也结合了场景拓扑传播特征、先验知识、实时事件序列进行了表示学习和故障预测。本文完成的工作可分为以下4点:

(1)本文提出了从硬件、软件、日志、指标时序数据等多源异构数据中自动构建组件-事件知识图谱的方案。其中,在事件因果关系判别模块,本文拓展事件特征至6种,有效解决了以往工作中需要人工反复标注的问题。

(2)本文针对组件-事件知识图谱设计了动态的知识表示学习模型。动态的实体表示充分地适应了运维场景特性,即同一事件实体在不同上下文中出现,有着不同的触发逻辑,对应的故障类型也会有所不同。

(3)本文引入知识图谱进行故障预测,提高了预测的准确率、细粒度和可解释性。本文的故障预测模型不仅可以预测是否会有故障产生,还会更细粒度的预测会有何种故障发生,且由知识图谱给出故障的触发逻辑。

(4)基于以上工作,设计了面向 IT 运维的可视化查询分析及故障预测系统。该系统有效地辅助运维人员实时查看系统运行状态,并由故障预测结果及时采取防范措施。

\section{未来工作展望}
本文深入调研IT运维难点及现有工作不足,提出了结合先验知识、场景实时特征的基于知识图谱的IT运维辅助方案。另外,本文所作工作仍有继续优化的空间。

(1)需要在工业界项目上进一步验证效果。目前本系统已在开源的分布式应用train-ticket和sock-shop上取得了良好效果。但工业界应用如淘宝、饿了吗、滴滴等,相较本文使用的两个开源应用,在客户请求量、系统复杂度上都要高出较多量级。因此,本文需要寻求在工业级应用中进一步验证,并采取相对应的优化措施。

(3)自动生成新的知识图谱,拓展知识库。本文虽然已模拟了常见的十余种故障,但仍难以涵盖实际运行中可能出现的所有故障,需要自动收集过往数据中未出现的新故障并沉淀生成对应的组件-事件知识图谱。