各位老师下午好,我今天进行答辩的论文题目为基于知识图谱的IT运维辅助系统设计与实现。
\par
在接下来我将分为5块内容展开介绍。包括研究背景与现状,研究目标与内容、研究方法与成果、系统实现与测试、总结与展望。

研究背景与现状
近年来,由于云计算灵活、便捷、可扩展的特性,越来越多电子商务、社交网络、金融、医药等领域的企业都选择将其服务应用部署于云计算环境中。在云计算环境中,这些服务应用会被部署在复杂庞大、跨协议的分布式集群上。分布式集群中实时数据不仅多源异构,且数量庞大。如下图分布式集群信息关系示意图所示,分布式集群中硬件、软件间有着交互依赖关系,同时也会实时产生各种数据,如曲线类的cpu,内存、io等指标时序数据,文本类的日志数据。组件数量量级为千,而数据的量级每秒可达到万。另外,分布式集群的技术、软件、配置不断演变,难以避免会出现各种故障。

为了防止和应对集群出现故障,云服务商需要运维人员进行IT运维工作。IT运维任务包括:全面获取集群各个粒度的运行状态信息;分析运行状态数据的含义及关系;根据分析结果,采取措施,预防应对故障的发生。依靠人工的IT运维面临着巨大挑战:运维人员难以快速查看每秒产生的上万条数据;运维人员不一定具备丰富的运维经验;运维人员不能迅速分析出结果。

为了应对人工运维的难点,近些年智能运维的概念被提了出来。智能运维(AIOps)可定义为任何模拟运维人员行为的计算机技术,目的是引入人工智能技术,提高IT运维效率,具体方式就是训练各种机器学习模型,用于完成指定的运维任务。目前已有的智能运维工作,仅仅依靠历史数据训练各种特定运维任务的模型,却忽略了其中蕴含的运维知识。运维知识能解释故障的产生过程,还具有通用性能够指导运维。由于已有智能运维工作缺少对运维知识的沉淀、表示和运用的研究,目前IT运维依然存在着三大问题:多源异构数据难以整合,运维知识表示不足,故障难以准确预知。

首先,针对多源异构数据难以整合的问题,本文对运行状态检测模型展开了调研。目前运行状态检测模型可以分为基于系统拓扑图的方法,基于事件因果图的方法,和基于运维知识图谱的方法。
基于系统拓扑图的方法,通过构建组件拓扑图,随后在该图上使用传播算法,帮助运维人员发掘重要的异常点。这种方法至关注了组见拓扑,忽略了日志、曲线异常之间的关系
