\chapter{绪论}
\section{研究背景与意义}
近年来,由于云计算灵活、便捷、可扩展的特性,越来越多电子商务、社交网络、金融、医药等领域的企业都选择将其服务应用部署于云计算环境中。在云计算环境中,这些服务应用会被部署在复杂庞大、跨协议的分布式集群上。随着分布式集群中的软硬件变迁、技术升级、配置更新等,各种故障时有发生,如资源争用死锁、软件运行错误、硬件故障等。另外,分布式集群结构的复杂性和资源的共享关联性会形成阻力,导致运维人员很难快速准确地查询分析分布式集群数据,进而导致故障未被及时发现而进一步扩散,最终导致应用不可用,带来巨大的经济损失。例如,云提供商AmazonWeb Service(AWS)在2015年9月20日上午2:13到7:10之间发生了一次服务宕机,当时用户无法访问Netflix,Tinder,Airbnb,Reddit和IMDb等流行的在线服务,失败的根源在于美国东部地区亚马逊DynamoDB服务的读写操作异常。

因此,如何借助人工智能技术辅助运维人员自动检测、分析、预测故障成为了研究的热点及难点。近些年被提出的智能运维(AIOps)就尝试将人工智能技术与运维相结合,通过机器学习的方法来提升运维效率。但大量已有AIOps方案均使用历史数据训练各种机器学习模型,用于完成异常检测、异常分类和故障预测等任务,却忽略了历史数据中蕴含的运维知识。这些运维知识,不仅能够解释故障产生过程,还具有通用性能够指导运维。由于缺少对运维知识的研究,AIOps仍然无法解决IT运维所面临的诸多问题。

% 运维知识,对于云服务提供商,可以为客户提供更稳定的服务以增加收益;对于运维人员,可以更直观,更有效地管理分布式集群减少人力消耗,并做出可解释、可靠的运维决策;对于云应用程序开发人员,可以减少应用程序部署,迭代和发布的阻力。

首先,如何全面地整合多源异构数据,获取运维知识始终是IT运维面临的一个问题。分布式集群中的数据不仅形式复杂多样,而且数量庞大。如图\ref{cloud-environment}所示,分布式集群中可用的数据包括组件间的拓扑关系、硬件的指标时序数据、软件的运行日志数据。组件间的拓扑关系指明了组件间的交互、依赖等关系,如负载均衡器(Server Load Balancer,SLB)会调控多台云服务器(Elastic Compute Service,ECS)实现负载均衡;硬件的指标时序数据记录了硬件的内存占用率、读写速度、CPU负载率等随时间的变化曲线;软件的运行日志数据记录了各个微服务在具体时间戳打印的日志文本。随着分布式集群的实时运转,平均每秒钟产生的各类数据可达上万条。

已有的运行状态监测模型\cite{wang2019grano,nie2016mining-causality-graph,qiu2020causality-mining-knowledge-graph}可以实时监测、收集分布式集群中的数据,并进一步沉淀运维知识。但其始终无法整合全部数据,都只监测到了片面的数据,导致获取到的运维知识也是片面的。现有运行状态监测模型的覆盖率、细粒度都有待提高,这样才能提升运维知识的全面性和通用性。
% 在已有的研究中,\parencite{wang2019grano,nie2016mining-causality-graph,qiu2020causality-mining-knowledge-graph}分别基于系统拓扑图、基于事件因果图、基于运维知识图谱监测集群运行状态,不断收集数据,并从中获取运维知识。但这些方法所监测到的数据都是片面的,导致获取到的运维知识也是片面的。系统拓扑图只关注组件间的交互关系,忽略了异常数据间的触发关系;事件因果图只关注集群异常事件之间的因果关系,忽略了事件所在的组件信息;运维知识图谱只关注数据种类之间的依赖关系,忽略了具体数据间的触发逻辑,构建的知识图谱细粒度较低。可见,现有运行状态监测模型的覆盖率、细粒度都有待提高,这样才能提升运维知识的全面性和通用性。

另外,如何有效地表示运维知识,也是IT运维一个亟待解决的问题。文献\parencite{nie2016mining-causality-graph,qiu2020causality-mining-knowledge-graph,tan2012prepare}将运维知识拓扑视作贝叶斯网,随后利用贝叶斯推理辅助运维工作。但贝叶斯网建立于条件独立性假设上,其认为给定父节点后,每个子节点与其非后代节点条件独立。事实上,在IT运维场景中,条件独立性假设是不成立的,比如故障产生与已发生的多条异常信息、系统整体状态都息息相关。这种运用贝叶斯网建模表示运维知识的方式,对IT运维辅助工作有一定提升,但局限于知识的显示结构,忽略了运维知识的深层逻辑。运维知识的表示,仍具有巨大提升空间。

IT运维还面临着另一个问题是如何准确预知未来可能会出现的故障。故障预测是辅助IT运维的代表性工作,其通过预测可能会发生的故障,提醒运维人员及时采取预防措施。当前,基于深度学习的故障预测方法\cite{xu2016health,cheng2018machine,du2017deeplog,das2018desh,islam2017predicting,li2020predicting,gao2020task}是研究的热点,其可以在故障预测任务上达到很好的预测效果,如文献\parencite{gao2020task}可以达到87\%的准确度和86\%的F1值。但是该类方法只能预测到是否会有故障发生,不能具体到会出现何种故障。另外,由于没引入运维知识,也不能给出故障触发过程,缺乏可解释性,可靠性低。在故障预测中引入运维知识,提高预测结果细粒度,增强预测可解释性,是目前亟待展开的工作。

\begin{figure}[htbp]
    \centering
    \includegraphics[width=.8\textwidth]{cloud-environment.png}
    \caption{分布式集群信息关系示意图\label{cloud-environment}}
\end{figure}

% 基于以上分析,首先本文分析了现有的分布式集群运行状态监测方案,发现这些方案仅使用了分布式集群部分信息且挖掘的数据特征有限,导致构建的监测拓扑图比较片面且其构建过程需要大量人工。本文收集了横跨软硬件的所有信息,并设计了6种数据特征,构建了组件-事件知识图谱。其次,本文针对云计算场景下异常沿组件图传播、事件发生与上下文相关的特性,提出了组件-事件知识图谱实体的动态表示方案。最后,本文用双向记忆网络编码事件序列,再用组件-事件知识图谱识别关键信息,从而将运维知识引入到故障预测中,增强了预测结果的可靠性与可解释性。
综上所述,目前在IT运维领域存在着以下挑战:

(1)在沉淀运维知识时,存在多源异构数据难以整合的挑战,已有的运行状态监测模型从不同方面获取集群运行状态信息,但都具有片面性,存在着巨大的提升空间。

(2)在表示运维知识时,存在运维知识表示不足的挑战,基于贝叶斯网建模表示的方式局限于显示的知识拓扑结构,需要进一步展开针对运维知识的表示学习研究。

(3)在故障预测时,存在着故障难以准确预知的挑战,只依靠深度学习的故障预测方法预测结果细粒度低,缺乏可解释性,需要展开引入运维知识进行故障预测的研究。

(4)针对上述的挑战,需要实现一种满足运维人员需求的IT运维辅助系统。该系统需要能够监测多源异构数据,有效地沉淀、表示运维知识,并用于实时的故障预测。

基于以上背景信息,本文收集了横跨软硬件的所有信息,构建了组件-事件知识图谱。其次,本文提出了组件-事件知识图谱实体的动态表示方案。随后,本文用双向记忆网络编码事件序列,再用组件-事件知识图谱识别关键信息,进行故障预测。最后,本文设计并实现了一个基于知识图谱的IT运维辅助系统。

\section{国内外研究现状}
为了解决IT运维实际中面临的问题,包括多源异构数据难以整合、运维知识表示不足和故障难以准确预知,国内外已经展开了大量研究。本文主要从分布式集群运行状态监测模型、知识表示学习和故障预测三方面展开了调研,并进行了分析。

\subsection{运行状态监测模型研究现状}
有效的分布式集群运行状态监测模型可以全面完善地获取、表示系统运行状态。高细粒度的运行状态数据是展开进一步运维工作的基础。

已有的模型分为基于系统拓扑图的方法、基于事件因果图的方法和基于运维知识图谱的方法。在基于系统拓扑图的方法中,eBay提出的GRANO\cite{wang2019grano}通过构建组件拓扑图,捕获警报信息和程序运行状态,实现了用于云计算分布式数据平台的端到端状态检测分析系统。具体实现上,GRANO首先处理大量指标时序数据来检测硬件和软件系统组件的异常信息,随后让专家利用领域知识给异常信息打上严重分数,然后在组件拓扑图上使用传播算法对各个组件的严重分数进行更新,最终辅助运维人员通过交互界面沿着系统拓扑图查看严重的系统组件。这种基于系统拓扑图的方法虽可以有效监视和识别系统异常,但不能沉淀出运维知识,也不能给出可靠的异常触发链。在基于事件因果图的方法中,文献\parencite{nie2016mining-causality-graph}首先利用异常检测方法\cite{nguyen2011pal}获取各个系统组件实时运行发生的异常事件,随后借助有监督的机器学习\cite{breiman2001randomforest}构建事件间可靠的因果关系,最终形成可辅助运维的事件因果图。这种基于事件因果图的方法不需要了解云计算场景中应用的设计和实现细节,也不需要检测服务的源代码,但只使用概率特征的因果发掘模型需要领域专家循环标注海量数据才可以达到良好的效果,且缺乏对系统组件拓扑关系的有效利用。在基于运维知识图谱的方法中,文献\parencite{qiu2020causality-mining-knowledge-graph}根据组件间的访问与部署关系、指标与组件间的来源产生关系,构建了运维知识图谱,但对于异常信息间的触发关系,该图谱只能精确到指标时序类型之间的因果关系,如“Container 1 CPU usage”导致了“Microservice 1 Response Time”\cite{qiu2020causality-mining-knowledge-graph},不能进一步知道什么样的CPU变化导致了什么样的响应时间变化。

可见始终缺乏一个横跨硬件、软件、异常数据,并能在高细粒度上将异常信息触发链和系统状态作为知识沉淀下来的运行状态监测模型。

\subsection{知识表示学习研究现状}
在运行状态监测模型沉淀出知识后,需要利用知识表示学习将运维知识转为计算机可理解的数据形式,即利用计算机可以处理的实值向量来表示知识实体和关系。运维知识被转化为实值向量后,可以通过计算机引入到下游任务中(如故障预测),有助于实现自动化运维。

知识表示学习方面,已经存在着大量工作。基于距离的模型中,文献\parencite{bordes2012joint}只利用了头实体和尾实体的共现信息来进行表示学习;在此基础上文献\parencite{bordes2011learning}将实体关系建模为两个分别针对头尾实体的矩阵,从而引入了关系信息。从翻译视角展开工作的模型中,最开始TranE\cite{bordes2013translatingE}将每个知识三元组中的关系视作头实体到尾实体的翻译过程,但它不能处理复杂的一对多、多对多关系;TransH\cite{wang2014knowledge}选择将头尾实体都映射到关系所在的超平面上,解决了复杂关系的问题;TransR\cite{lin2015learning}认为每一种关系间是有区别的,应该对应着不同的语义向量空间,所以其将关系嵌入为矩阵;TransD\cite{ji2015knowledge}为了克服同一关系头尾实体种类属性差别过大的问题,将每个对象(实体、关系)嵌入为语义向量和映射向量这两个向量;TranSparse\cite{ji2016knowledge}根据关系连接的头尾实体对数量,自适应的调整映射矩阵的稀疏程度;TransF\cite{feng2016knowledge}放宽了损失函数的约束条件,只要求头实体加关系后的张量与尾实体关系一致;TransG\cite{ou2016asymmetric}将实体的不确定度、关系的多语义用高斯分布协方差来表示。基于语义匹配的模型中,LFM\cite{jenatton2012latent}使用关系的双线性变换,刻画实体和关系之间的二阶联系;ComplEx\cite{trouillon2016complex}为了建模非对称关系将表示扩展到复数向量空间;ANALOGY\cite{liu2017analogical}将知识图谱中的类比关系进行了建模;SLM\cite{socher2013reasoning}开始引入神经网络,其利用标准非线性单层神经网络来连接实体;NTN\cite{socher2013reasoning}使用张量网络捕获头尾实体间的语义关联;ConvE\cite{dettmers2018convolutional}为了捕获实体间的语义关系,使用了卷积网络。融入多源信息的模型也有很多,SSE\cite{guo2015semantically}和TKRL\cite{xie2016representation}将实体的类别信息引入了知识表示学习中;文献\parencite{lin2015modeling}为了引入关系路径将路径上所有的关系向量进行了组合以表示路径向量;为了引入文本描述,文献\parencite{socher2013reasoning}将语料库训练的词向量累加平均作为实体的初始向量;文献\parencite{xie2016representation}给每个实体赋予了基于结构和基于文本描述的两个表示;文献\parencite{wang2016text}选择将实体与文本库词汇对齐;GAKE\cite{feng2016gake}和TCE\cite{shi2017knowledge}则引入了实体的上下文信息,包括邻居上下文、边上下文、路径上下文等。文献\parencite{schlichtkrull2018modeling}引入图卷积网络的同时区分了不同关系对节点的影响,利用图信息传播机制编码知识图谱以获取实体表示。

虽然知识表示学习方面已经存在了很多不同角度展开的工作,但都只能得到实体、关系的静态表示向量。在云计算场景下的组件-事件知识图谱中,实体会随着上下文场景变化有不同的含义,需要动态地表示。

\subsection{故障预测研究现状}
故障预测是一种主动预防故障的方法,可以在故障发生前进行预测,从而告知运维人员采取措施防止故障发生,避免故障造成的损失。
% 马尔科夫模型是一种广泛应用于时间序列数据建模的统计方法。作为有限时间马尔可夫链(DTMC)的扩展,隐马尔可夫模型(HMM)已经成功地应用于许多模式识别任务中,比如语音处理和基因序列分析。尽管基于隐马尔可夫模型的异常预测是近年来众多研究的主题,但要其满足商业应用需求(例如,高可扩展性和预测精度)仍然是一个挑战。

文献\parencite{salfner2010survey}已经对在线故障预测技术进行了广泛地调研。其将现有的故障预测方法宽泛地分为两类:分类方法和函数逼近方法。基于分类的故障预测方法\cite{tan2012prepare,pitakrat2018hora,zhang2018prefix,baldoni2015line}首先构建了容易发生故障和不容易发生故障的数据样本,随后在这些数据上训练分类器。分类器的决策边界通常来自于参考数据集,这个数据集每个数据点的决策边界都是已知的,即明确地知道它指示的是容易发生故障还是不容易发生故障。在进行在线故障预测时,只需要检查当前监视值位于决策边界的哪一侧即可。函数逼近是一个广泛应用于各种科学领域的术语,可以用在故障预测任务的原因在于该任务可以被视作发掘一种函数映射关系,即被监视的系统状态(函数的输入)到系统未来是否会有故障(函数的输出)之间的未知函数关系。逻辑回归技术就是为目标性能值建立曲线拟合函数,并调整参数以最佳拟合训练数据集,其输入数据一般是性能指标时序数据\cite{salfner2010survey}。文献\parencite{dalmazo2013predicting}提出了一种基于统计模型的流量预测方法,它将滑动窗口内的观测值用泊松分布加权。文献\parencite{sladescu2012event}提出了一种事件感知策略,通过利用与计划事件相关的先验知识,可以更有效地预测系统工作时的负载突发。事件感知预测方法可以非常有效地预测故障,但无法给出故障出现的触发过程。文献\parencite{purushotham2005multi}使用了基于隐马尔可夫模型的方法来推断受监控组件的状态是否健康。虽然作者文中未提及该方法可用于故障预测,但可以通过以下方式进行故障预测:假设系统中存在导致未来故障的错误状态,而其他错误状态不会导致未来的故障,所提出的隐马尔可夫模型方法就可以用来确定(分类)故障是否即将发生。文献\parencite{boutros2011detection}指出,可以通过识别系统正在经历的状态来预测系统出现故障的概率,但作者并没有给出详细的算法和实验结果。文献\parencite{xu2016health,cheng2018machine,du2017deeplog,das2018desh,islam2017predicting,li2020predicting}开始使用深度学习方法,如循环神经网络(RNN,Recurrent Neural Networks)和长短期记忆神经网络(LSTM,Long Short-Term Memory)进行故障预测。文献\parencite{gao2020task}使用双向的长短期记忆神经网络(Bi-directional Long Short-Term Memory,BiLSTM)编码时序信息,并对数据赋予了不同权重,在预测是否会发生故障时取得了最佳效果。

从以上调研结果可见,目前的故障预测工作专注于预测是否会有故障发生,不能预测到具体会有何种故障。另外,由于预测过程均没有引入运维知识,预测结果的可解释性不高。

\section{论文研究内容与目标}
\subsection{研究目标}
本文的研究目标是设计并实现一个满足实际运维需求的IT运维辅助系统。为了达到研究目标,本文分析了目前的研究难点,并逐一展开了调研,总结了现有工作存在的问题。在此基础上,本文提出了一个基于知识图谱的IT运维辅助系统的设计方案。该方案基于本文提出的算法模型,将知识图谱构建、知识图谱表示学习、故障预测封装为多个功能模块,以应对多源异构数据难以整合、运维知识表示不足和故障难以准确预知等问题。

\subsection{研究内容}
本文主要开展了以下研究工作:
\begin{itemize}
    \item [(1)]研究横跨硬件、软件、异常数据的组件-事件知识图谱自动构建方案。一方面研究如何让知识图谱整合高细粒度的多种信息,如软硬组件间关系、指标时序数据、日志数据。另一方面研究如何设计事件特征,使用机器学习模型挖掘事件间因果关系。最终实现从历史数据中沉淀出故障类型对应的组件-事件知识图谱。
    \item [(2)]研究针对组件-事件知识图谱的知识表示学习模型。在组件-事件知识图谱中,异常会沿着组件拓扑关系传递,事件间有因果触发关系,实体在不同的上下文中也有不同的含义。针对组件-事件知识图谱的这些特性,需要重点研究如何让实体随上下文变化动态地表示。
    \item [(3)]研究基于组件-事件知识图谱的故障预测模型。为了解决深度学习模型在故障预测时,可解释性低、预测结果细粒度低的问题,需要研究如何结合知识图谱与深度学习模型进行故障预测。最终使得预测结果可以具体到会出现何种故障,并能结合知识图谱得到触发逻辑。
    \item [(4)]研究基于知识图谱的IT运维辅助系统的设计与实现。为了满足运维人员使用需求,需要研究如何整合业务流程。重点研究如何实现分布式集群不同信息的实时快速获取,以及如何向运维人员展示不同的运行状态信息。
\end{itemize}

\section{论文结构安排}
本文篇幅结构共有七章,各个章节间的关系如图\ref{paper-constructure}所示,每一章内容如下:

第一章为绪论,首先梳理了相关研究背景和研究现状,然后分点描述了本文研究内容。

第二章为背景知识,描述了本文会涉及到的相关技术,为下文各个方法模型的提出做好铺垫。

第三章为组件-事件知识图谱的构建流程,主要介绍了系统运行状态信息获取,事件因果关系挖掘和历史数据沉淀生成组件-事件知识图谱的过程,在模拟构建的数据集上进行了实验,并分析了实验结果。

第四章为组件-事件知识图谱的知识表示学习,主要分析了场景特性,介绍了随上下文变化的动态表示学习方案,在三元组分类和链接预测任务上进行了对比实验,并分析了实验结果。

第五章为引入组件-事件知识图谱的故障预测,主要介绍了引入组件-事件知识图谱,结合实时事件序列做出可解释故障预测的方法,在事件序列故障预测任务上进行了对比实验,并分析了实验结果。

第六章为系统设计与实现,主要介绍基于本文方法实现了一个基于知识图谱的IT运维辅助系统,并详细介绍该系统的设计和实现细节。

第七章为总结与展望,对本文完成的工作进行了总结归纳,也对未来的工作进行了规划。
\begin{figure}[htbp]
    \centering
    \includegraphics[width=.8\textwidth]{paper-constructure.pdf}
    \caption{论文章节关系示意图\label{paper-constructure}}
\end{figure}

% Definition of rule: E is symptom events unit, A; B 2
% E; A ! B means A causes B and A happens before B. !
% presents the causality.


% 详见文献\cite{Peebles2001-100-100}\parencite{Babu2014--}
% 参考文献\cite[见][49页]{于潇2012-1518-1523}\parencite[见][49页]{Babu2014--}
% 硕士论文\cite{zhouGPS2015},博士论文\cite{余勇1998--}
% 大小写\cite{liu_statistical_2017}

\nomenclature{IT}{Information Technology}
\nomenclature{Ops}{Operations}
\nomenclature{KPI}{Key Performance Indicators}
\nomenclature{AIOps}{Artificial Intelligence for IT Operations}
\nomenclature{API}{Application Programming Interface}

% 如图\ref{lxfbook}所示。

% \begin{figure}
%     \centering
%     \includegraphics[width=.6\textwidth]{lxfbook.jpg}
%     \caption{陆小凤传奇\label{lxfbook}}
% \end{figure}
